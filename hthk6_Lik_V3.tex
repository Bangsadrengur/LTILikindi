%        File: hthk6_Lik_V3.tex
%     Created: Wed Apr 25 07:00 PM 2012 G
% Last Change: Wed Apr 25 07:00 PM 2012 G
%
\documentclass[11pt,a4paper]{article}
\usepackage[icelandic]{babel}
\usepackage[T1]{fontenc}
\usepackage[utf8]{inputenc}
\usepackage{enumerate}
\usepackage{amsmath}
\usepackage{amsfonts}
\usepackage{fullpage}
\usepackage{graphicx}
\author{Nemi: Heimir Þór Kjartansson\\Verkefnakennari: Jakob Sigurðsson}
\title{Fag: Líkindaaðferðir verklegt \\ Skiladæmi 3}
\begin{document}
\maketitle
\section{Inngangur}
Hér er skoðað LTI kerfi og hegðun þess við slembið inntak. Við umfjöllun
og úrvinnslu er mikið stuðst við aðferðir og niðurstöður úr kennslubók
áfangans. Verkefnið er unnið sem þriðja verklega verkefni fyrir áfangann
Líkindaaðferðir við Háskóla Íslands.
\section{LTI kerfi}
Unnið er með LTI kerfi þar sem inntak er sett í gegnum Butterworth síu
og titlum við yfirfærslufall hennar $H(f)$. 
Aflyfirfærslufall kerfisins hefur eiginleikana
\begin{eqnarray*}
    |H(f)|^2 = \frac{1}{1+\left[\frac{f^2-f_uf_l}{f(f_u-f_l)}\right]^2}
\end{eqnarray*}
þar sem mörk síunnar eru $f_u=110Hz$ og $f_l=90Hz$.
\section{Inn- og úttak kerfisins} 
Inntakið sem skoðað er, $X(t)$, er vítt staðnað slembiferli. Skoðað 
verður úttak kerfisins, $Y(t)$.

Gefið er sjálffylgnifall innmerkisins
\begin{eqnarray*}
    R_X(\tau)=5^{-600|\tau|}
\end{eqnarray*}

Tíðnirófþéttleiki úttaksins er fundinn og settur fram sem jafna
(8-83) í námsbók og má sjá frekari atriði útleiðslu þar.
\begin{eqnarray*}
    S_Y(f)=\frac{6.0792\cdot 10^4 f^2}{f^6 -1.02811\cdot 10^4 f^4
    -7.88969\cdot 10^7 f^2 +8.93744 \cdot 10^{11}}
\end{eqnarray*}

Sýni úr inntaki má finna með lítilvæglegri umritun á M-skrá úr 
námsbók, þ.e. ,sysxmp5.m'. Sjá ???? fyrir kóða.
?????MYND??????
\section{Föll sem finna rófsþéttleika} \label{se:foll}
Í kóðahluta ???? eru tilgreind þrjú föll sem hvert finnur 
rófsþéttleika. 
\begin{verbatim}
[Shat] = spect_est_pg(x, dt)
\end{verbatim}
tekur inn sýniferli $x$ og söfnunarbil $dt$ og skilar rófþéttleika 
sýniferlisins $Shat$.

\begin{verbatim}
[Shat] = spect_ext_ac(x, dt, M)
\end{verbatim}
tekur inn sýniferli $x$, söfnunarbil $dt$ og tímaseinkun $M$ og skilar
rófþéttleika sýniferlisins $Shat$.

\begin{verbatim}
[Shat] = spect_est_x(x, dt, wtype, stype)
\end{verbatim}
tekur inn sýniferli $x$, söfnunarbil $dt$, gluggategund $wtype$ þar 
sem 1 er boxcar, 2 er hamming og 3 hanning, og að lokum $stype$ 
eða róftýpu þar sem 1 er nákvæmni í tind og 2 er mýkt fall. Þá skilar
fallið rófþéttleika $Shat$.
\section{Meðalkvaðratskekkja} \label{se:mse}
Sýna skal að meðalkvaðratskekkju $MSE$ megi setja fram sem línulega
samantekt annars veldis bjögunar metna rófþéttleikans og ferviki hans.

Það er 
\begin{align}
    MSE &= E[(S_X(\omega)-\hat{S}_X(\omega)^2] \label{eq:lok}\\
    &= (S_X(\omega)-E[\hat{S}_X])^2+E[(\hat{S}_X(\omega)-
    E[\hat{S}_X(\omega)])^2] \label{eq:tilbaka} \\
    &= bias(\hat{S}_X(\omega))^2+var(\hat{S}_X(\omega))
\end{align}

Hér er jafna (\ref{eq:tilbaka}) umrituð þar til komið er á form
jöfnu (\ref{eq:lok}) og niðurstaðan þar sem sönnuð. Þar sem öll föll eru 
föll af horntíðni $\omega$ er sleppt að rita það fyrir hvert fall.
Einnig er sleppt að taka fram hvaða ferli fallið tilheyrir, þau 
tilheyra öll saman ferli $X$.
\begin{align*}
    (S-E[\hat{S}])^2 + E[(\hat{S}-E[\hat{S}])^2] \\
    &= S^2 - 2SE[\hat{S}]+E[\hat{S}]^2
    + E[\hat{S}^2-2\hat{S}E[\hat{S}]+E[\hat{S}]^{2}] \\
    &= E[S^2]-2E[S]E[\hat{S}] +2E[\hat{S}]^2 
    +E[\hat{S}^2]-2E[\hat{S}E[\hat{S}]] \\
    &= E[S^2-2S\hat{S}+\hat{S}^2]+2E[\hat{S}]^2-2E[\hat{S}E[\hat{S}]] \\
    &= E[(S-\hat{S})^2] + \varphi \\
    \varphi &= 2E[\hat{S}]^2-2E[\hat{S}E[\hat{S}]]
\end{align*}
Sýna má að $\varphi=0$ en til þess skal fyrst skoða seinni lið $\varphi$.
\begin{align*}
    E[\hat{S}E[\hat{S}]] \\
    &= \frac{1}{N}\sum_N \hat{S} E[\hat{S}] \\
    &= \frac{1}{N}\sum_N \hat{S} 
    \left(\frac{1}{N} \sum_N \hat{S}\right) \\
    &= \frac{1}{N^2}\left(\sum_N \hat{S} \right) ^{2} \\
    &= E[\hat{S}]^2
\end{align*}
þar sem $N$ er fjöldi staka í $\hat{S}$.

Þá má ljóslega sjá að liðir $\varphi$ eyða hvorum öðrum út svo 
$\varphi=0$ og jafna (\ref{eq:lok}) er jöfn jöfnu (\ref{eq:tilbaka})
og niðurstaðan heldur.
\section{Samanburður aðferða}
Notuð er fyrrgreind aðferð, sem nú er sýnt að virki skv. \ref{se:mse} 
hluta, til að meta föllin úr \ref{se:foll} hluta. Niðurstöðurnar
má lesa úr töflu \ref{tab:tafla}. Gildi úr líkani sem borið er saman
við er tíðnigildi er lögleg Nyquist tíðni eða 
$f_N = 2\cdot2\cdot\pi\cdot f_s$.
\begin{table}[htbp]
    \centering
    \begin{tabular}{l | l | l | l}
        Fall / gluggatýpa & Fervik & Bjögun & MSE \\ \hline
        spect\_est\_pg    & 2.04E-06 & 6.09E-08 & 2.04E-06 \\
        spect\_est\_ac    & 2.87E-07 & 6.17E-08 & 2.87E-07 \\
        spect\_est\_x hamming & 2.92E-03 & 8.73E-04 & 2.92E-03 \\
        spect\_est\_x hanning & 3.16E-03 & 1.06E-03 & 3.16E-03 \\
    \end{tabular}
    \caption{Mat á rófþéttleikaföllum.}
    \label{tab:tafla}
\end{table}
\section{Kóði}
LikVerk3.m
\begin{verbatim}
% Útdráttur úr sysxmp5.m úr bók
%% b)
fs=2000;
dt=1/fs;
t2=0:dt:0.016; % 0.016 / dt + 1 = 33 punktar 
hn=exp(-600*t2);
hn(1)=0.5*hn(1); % Helmingsvægi endapunkta
randn('seed',500);
x=sqrt(6000*2000)*randn(1,2000);
n1=dt*conv(hn,x); % Sýni úr inntaki
t=0:dt:(size(n1)(2)-1)*dt;
figure(1);
plot(t,n1);
xlabel('t'); ylabel('X(t)');
%% c)
n=n1(33:1999);
a1=126.28; a2=-62.83; a3=622; a4=-5.76*2*pi/180;
t1=0:dt:0.08;
h=a1*exp(a2*t1).*cos(a3*t1+a4*ones(size(t1)));
y1=dt*conv(h,n);
t=0:dt:(size(y1)(2)-1)*dt;
figure(2);
plot(t,y1);
xlabel('t'); ylabel('Y(t)');
%% d) 1. 
S1=spect_est_pg(y1,dt);
tau1=(length(S1)-1)/2;
figure(3);
plot([-tau1:tau1]./10,S1);
xlabel('f');
ylabel('S_y(f)');
%% d) 2.
% M sett á gildi 16 þar sem slíkt gildi
% er notað í dæmi í námsbók.
S2=spect_est_ac(y1,dt, 16);
tau2=(length(S2)-1)/2;
figure(4);
plot([-tau2:tau2].*1.75,S2);
xlabel('f');
ylabel('S_y(f)');
%% d) 3. Hamming
S3=spect_est_x(y1,dt, 2, 1);
tau3=(length(S3)-1)/2;
figure(5);
plot([-tau3:tau3].*3.7,S3);
xlabel('f');
ylabel('S_y(f)');
%% d) 3. Hanning
S4=spect_est_x(y1,dt, 3, 1);
tau4=(length(S4)-1)/2;
figure(6);
plot([-tau4:tau4].*3.7,S4);
xlabel('f');
ylabel('S_y(f)');
%% Samanburður Hamming og hanning.S3 er Hamming, S4 er Hanning.
figure(7);
plot([-tau4:tau4].*3.7,S4-S3);
xlabel('f');
ylabel('Hanning Sy - Hamming Sy');
%% Fræðilegt S_y við nyquistgilda söfnunartíðni.
fN=2*(2*pi*fs);
Sy=(6.0792*10^4).*fN.^2./(fN.^6-1.01811.*10.^4.*fN.^4 ...
+7.88969.*10.^2.*fN.^2+8.93744.*10.^4);
bias1=(Sy-mean(S1)).^2
var1=var(S1)
mse1=bias1.^2+var1;
mse1=mean(mse1)
bias2=(Sy-mean(S2)).^2
var2=var(S2)
mse2=bias2.^2+var2;
mse2=mean(mse2)
bias3=(Sy-mean(S3)).^2
var3=var(S3)
mse3=bias3.^2+var3;
mse3=mean(mse3)
bias4=(Sy-mean(S4)).^2
var4=var(S4)
mse4=bias4.^2+var4;
mse4=mean(mse4)
\end{verbatim}

spect\_est\_pg.m
\begin{verbatim}
function [Shat] = spect_est_pg(x, dt)
Fx = fft(x);
absSquare = ((abs(Fx)).^2)./length(x);
Shat = dt.*absSquare;
\end{verbatim}

spect\_est\_ac.m
\begin{verbatim}
function [Shat] = spect_est_ac(x, dt, M)
% Umritun á corspec úr kafla 7-9
fs=1/dt;
[a,b]=size(x);
if a<b % Gera inntak að dálkvigur
    x=x';
    N=b;
else
    N=a;
end
x1=detrend(x,0); % Fjarlægja fastaþátt.
x1(2*N-2)=0;
R1=real(ifft(abs(fft(x1)).^2));
W=triang(2*N-1);
R2=[R1(N:2*N-2);R1(1:N-1)]./((N)*W(1:2*N-2));
R3=R2(N-M:N+M-1);
H=hamming(2*M+1);
R4=R3.*H(1:2*M);
k=2^(ceil(log2(2*M))+2);
S1=abs((1/fs)*fft(R4,k));
Shat = S1;
\end{verbatim}

spect\_est\_x.m
\end{document}


